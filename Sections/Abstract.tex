%----------------------------------------------------------------------------------------
%	ABSTRACT PAGE
%----------------------------------------------------------------------------------------

\begin{abstract}
\addchaptertocentry{\abstractname} % Add the abstract to the table of contents
As the number of in-orbit satellites increases, the need for a way to service them becomes increasingly critical.\\
Recently the EU funded EROSS, a project with the purpose of providing a new range of services for in orbit satellites with consequent analysis for satellite design and life-cycle management.
This initiative aims to enhance the availability of cost-effective and secure orbital services by assessing and validating the essential technological components of the Servicer spacecraft. The incorporation of robotic space technologies working on this project will lead to greater autonomy and safety in executing these services in space, requiring reduced ground-based supervision.

This master's thesis presents an innovative approach to pose estimation using deep learning and computer vision techniques. The research explores the development and implementation of a system for in-orbit satellites pose estimation. Delving into the complexities of rendezvous maneuvers, the system devised herein addresses the challenges associated with achieving and maintaining accurate pose estimations in the ever-changing and demanding conditions of space. Through a comprehensive exploration, this thesis contributes valuable insights and practical solutions to enhance the reliability and efficiency of satellite rendezvous processes.

A mono camera system is employed, reducing the hardware complexity and costs while maintaining performance. The camera captures pictures of the target satellite during the whole approach phase.
A deep learning framework, based on a Convolutional Neural Network (CNN), is used to identify and track landmark features on the target satellite from captured images. This CNN-based approach provides high accuracy in feature recognition and tracking precision.
A neural network-based regression model is introduced to map the 2D image coordinates or identified landmarks to their corresponding 3D coordinates with respect to the camera frame. This implementation permits to have a mono-camera instead of a stereo-camera system.
Finally, incorporating the CPD algorithm, the system aligns the predicted 3D point clouds to the reference model, enabling accurate pose estimation and tracking.

The proposed system is tested through simulations. The results demonstrate the system's capability to estimate the pose of in-orbit satellites. This research contributes to the advancement of autonomous satellite operations, space debris management, and space exploration. Furthermore, it has the potential to enhance satellite rendezvous and capture capabilities.
\end{abstract}