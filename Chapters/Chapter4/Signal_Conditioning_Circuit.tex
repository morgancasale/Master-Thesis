\section{Signal Conditioning Circuit}
\label{Signal_Conditioning_Circuit}
We know that the controller DAC outputs a voltage between 0 and 3.3V, the power stage has a gain of 10, and that to drive Flexar's coils, at their rated maximum power of 0.8W, we need to provide a voltage of about 6V at a current of 0.2A. \\
A very simple solution is to implement a variable voltage divider to adjust the amplitude of the signal coming from the DAC.
We chose a maximum dividing factor of 10 to match the power stage gain.

\begin{figure}[!ht]
    \centering
    \resizebox{.7\linewidth}{!}{\begin{circuitikz}[american]
    \draw (0,0) node[ground]{} to[sV, l=$V_{DAC}$] (0,3) to[vR, l=$R_1$] (3,3) to[R, l=$R_2$] (3,0) node[ground]{};
    \draw (3,3) to[short, -o] (4,3) node[right]{$V_{out}$};
\end{circuitikz}}
    \caption{Signal conditioning circuit.}
    \label{fig:cond_circuit}
\end{figure}

Where: 
\begin{itemize}
    \item $V_{DAC}$ is the voltage coming from the controller DAC [0,3.3]V.
    \item $R_1$ is a 100$k\Omega$ potentiometer to adjust the amplitude of the signal.
    \item $R_2$ is a 10$k\Omega$ resistor to set the maximum amplitude of the signal.
    \item $V_{out}$ is the output voltage of the conditioning circuit [0, 0.33]V.
\end{itemize}

The output voltage of the conditioning circuit is given by the following formula:
\begin{equation}
    V_{out} = V_{DAC} \cdot \frac{R_1}{R_1 + R_2}
\end{equation}

The values of $R_1$ and $R_2$ have been chosen to be 100$k\Omega$ and 10$k\Omega$ respectively, as they are standard values, provide a good range of adjustment for the amplitude of the signal, and their order is big enogh to work with the provided DAC current of 12mA.