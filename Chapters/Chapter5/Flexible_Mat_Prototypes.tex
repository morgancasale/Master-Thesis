\section{Flexible Mat Prototypes}
\label{sec: Flexible_Mat_Prototypes}
The goal of these prototypes was to create a small silicone mat where the magnet could be suspended in a membrane integrated into the mat itself.
The coil and its heatsink would also be "trapped" inside the mat.
As the mat would be made of silicone, it would allow the device to flex with the flexible coil.
As we can't 3D print silicone we had to create the entire design to be able to make it by silicone casting taking into account all the design limitations of this method.
All the pieces were designed in SolidWorks and then 3D printed to create the molds for the silicone casting. 

% -- Subsection 3.1
\subsection{Design of the membrane}
The design goal for the membrane was to create a structure that could support the magnet just enough to win over its gravitational force so that any other force applied to it on the z-axis would be enough to move it and make it vibrate.
For the membrane design, we decided to use a simple Celtic-cross structure as we described previously in section \ref{sec: Membrane-magnet_system}.
This resulted in a membrane with a central cylindrical chamber used to trap the magnet which is suspended by four parallelepipedal arms.
\begin{figure}
    \centering
    \resizebox{0.9\textwidth}{!}{
        \includegraphics{Chapters/Chapter5/Flexible_Mat_Prototypes/Figures/membrane_v1.png}
    }
    \caption{Membrane virtual model of the small magnet prototype.}
    \label{fig: Membrane_v1_model}
\end{figure}

\subsubsection{Material stiffness and thickness}
While prototyping we experimented with the same two different silicone materials we described in section \ref{sec: Sleeve_production}.
We quickly realized that the softer silicone was too soggy as the membrane arms would need to be too thick to support the magnet.
We so decided to move forward with only the harder silicone which allowed us to create thinner arms.

\subsubsection{Membrane structure vs magnet dimensions}
The main prototypes we realized were designed with two different N52 cylindrical magnets in mind.
The first one is a small 10mm diameter and 2mm thick magnet, the second one is a 15mm diameter and 3mm thick magnet.
The two magnets also have very different weights, the small one weighs 1.13g while the big one 4.03g.
The first design of the membrane was based on the small magnet, thanks to it being lightweight it could be supported by a membrane with very thin arms (0.6mm) and a width of 4mm.
The membrane arms needed also to be long enough to allow the magnet to move freely on the z-axis, so we decided to make them 4mm long.
\begin{figure}
    \centering
    \resizebox{0.9\textwidth}{!}{
        \includegraphics{Chapters/Chapter5/Flexible_Mat_Prototypes/Figures/membrane_v1_section.png}
    }
    \caption{Membrane cross-section of the small magnet prototype (t->thickness, L->length).}
    \label{fig: Membrane_v1_section}
\end{figure}

Switching to the big magnet we realized that the membrane arms would need to be thicker to support the magnet, even if we made them wider (5mm).
This was due to the increased weight of the magnet and the increase in the arms' length (now 5mm) we needed to make to allow the magnet to move freely on the z-axis.
To find the minimal thickness we used the model described in subsection \ref{sec: Membrane_stiffness}.
We set a maximum deflection of 0.8mm for the membrane arms and calculated the thickness needed to support the magnet using equations \ref{eq: Beam_deflection} and \ref{eq: Beam_inertia}.
The results showed that the membrane arms would need to be at least 1.45mm thick to support the magnet, so we decided to make them 1.6mm thick to have a safety margin.

The next problem we encountered arose when we observed that the membrane arms were breaking at the connection with the cylindrical chamber.
This was the abrupt change of profile that was causing a stress concentration at that point.
To solve this problem we decided to add a small fillet to the connection between the arms and the chamber.
\begin{figure}
    \centering
    \resizebox{0.9\textwidth}{!}{
        \includegraphics{Chapters/Chapter5/Flexible_Mat_Prototypes/Figures/membrane_v2_section.png}
    }
    \caption{Membrane cross-section of the big magnet prototype.}
    \label{fig: Membrane_v2_section}
\end{figure}

% -- Subsection 3.2
\subsection{Design of the mat}
The mat structure is based on a simple idea but its design was quite complex to be realized with silicon casting. This is mostly due to our goal of integrating the membrane into the mat itself.
This is due to our design goals:
\begin{itemize}
    \item The membrane and magnet need to be integrated into the mat structure.
    \item There needs to be a mechanical way to keep the coil and its heatsink steady inside the silicone structure of the mat, as nothing can be glued to silicone.
    \item We need to create a channel for the magnet chamber to move freely on the z-axis.
    \item The complete structure must be able to flex somewhat.
\end{itemize}

As the silicon sleeve of the previous prototype \ref{sec: Sleeve_production} we opted for a two-part mold to create the mat:
\begin{itemize}
    \item \textbf{Mold cavity: } The cavity was designed as a parallelepipedal empty box to create a simple parallelepipedal external structure for the mat.
    \begin{figure}
        \centering
        \resizebox{0.5\textwidth}{!}{
            \includegraphics{Chapters/Chapter5/Flexible_Mat_Prototypes/Figures/mold_cavity.png}
        }
        \caption{Flexible mat mold cavity}
        \label{fig: mat_mold_cavity}
    \end{figure}
    The purple component in figure \ref{fig: mat_mold_cavity} has four screw holes at the top to screw the suspended core to the cavity.
    The cavity also has a rectangular hole at the bottom where the small component in light blue is inserted.
    This component function is to work as a pedestal for the magnet, its exact function will be explained in the next section.

    Through both components are carved four holes, these are used to allow the excess silicone of the casting process to flow out of the cavity.

    The purple part was 3D-printed in a flexible material called TPU to allow the mat to be easily removed from the cavity.

    \item \textbf{Mold core: } The core is composed of multiple components that are inserted into the cavity to create the internal structure of the mat.
    \begin{figure} %TODO: Fix dimensions of the compose
        \centering
        \begin{subfigure}[b]{0.475\textwidth}
            \centering
            \resizebox{\textwidth}{!}{
                \includegraphics{Chapters/Chapter5/Flexible_Mat_Prototypes/Figures/mold_core_exploded_top.PNG}
            }
            \caption{Exploded top view of the mold core.}
        \end{subfigure}
        \hfill
        \begin{subfigure}[b]{0.475\textwidth}
            \centering
            \resizebox{\textwidth}{!}{
                \includegraphics{Chapters/Chapter5/Flexible_Mat_Prototypes/Figures/mold_core_exploded_btm.PNG}
            }
            \caption{Exploded bottom view of the mold core.}            
        \end{subfigure}
        \vskip\baselineskip
        \begin{subfigure}[b]{0.475\textwidth}
            \centering
            \resizebox{\textwidth}{!}{
                \includegraphics{Chapters/Chapter5/Flexible_Mat_Prototypes/Figures/mold_core_top.PNG}
            }
            \caption{Assembled top view of the mold core.}
        \end{subfigure}
        \hfill
        \begin{subfigure}[b]{0.475\textwidth}
            \centering
            \resizebox{\textwidth}{!}{
                \includegraphics{Chapters/Chapter5/Flexible_Mat_Prototypes/Figures/mold_core_btm.PNG}
            }
            \caption{Assembled bottom view of the mold core.}
        \end{subfigure}
        \caption{Assembly of the mold core}
        \label{fig: mat_mold_core}
    \end{figure}
    In figure \ref{fig: mold_core} we can see all the components of the core, which are:
    \begin{itemize}
        \item \textbf{Mold core center } (in light blue) \textbf{: } This is the central component of the core, it is used to create the coil membrane and the channel for the magnet chamber to move.
        \item \textbf{Coil trap} (in red) \textbf{: } This is the structure that will house the coil and its heatsink and the only part of the core that will remain inside the mat.
        \item \textbf{Mold core cap } (in yellow) \textbf{: } This component is used to keep the coil trap in place and to prevent the silicone from entering the coil trap.
        \item \textbf{Mold core - cavity bridge } (in pink) \textbf{: } This part screws into the core and the cavity to keep the core suspended in the cavity.
    \end {itemize}
    
\end{itemize}

\subsubsection{Magnet chamber and membrane}
To create the membrane and trap the magnet inside its chamber we needed a way to allow the silicone to flow all around the magnet.
\begin{figure}
    \centering
    \begin{subfigure}[b]{0.475\textwidth}
        \centering
        \resizebox{\textwidth}{!}{
            \includegraphics{Chapters/Chapter5/Flexible_Mat_Prototypes/Figures/mat_mold_core_center_top.PNG}
        }
        \caption{Mold core center top view.}
        \label{fig: mat_mold_core_center_top}
    \end{subfigure}
    \hfill
    \begin{subfigure}[b]{0.475\textwidth}
        \centering
        \resizebox{\textwidth}{!}{
            \includegraphics{Chapters/Chapter5/Flexible_Mat_Prototypes/Figures/mat_mold_core_center_btm.PNG}
        }
        \caption{Mold core center bottom view.}
        \label{fig: mat_mold_core_center_btm}
    \end{subfigure}
    \caption{Mold core center.}
    \label{fig: mat_mold_core_center}
\end{figure}
To create the upper part of the chamber we modeled the mold core center to have a cylindrical hole in its center, deep and large enough to create a silicone wall around the magnet with a lateral thickness of 0.5mm and a bottom thickness of 1.2mm.
To cast the upper part of the chamber we created a pedestal (the light blue component in figure \ref{fig: mat_mold_cavity}) for the magnet to rest on, the pedestal is a simple rectangular structure with a small circular bump at its center where the magnet is glued on.
When the silicone is set the pedestal can be removed and the magnet will be trapped inside the chamber by an upper silicone ceiling of thickness 0.6mm.
The curved parts and the square holes we can see in figure \ref{fig: mat_mold_core_center} are used to create the membrane arms.

The part used to create the channel is composed of three different-sized cylinders, the first one (the largest) is used to create the space for the membrane arms to flex, the second one is used to create the channel itself and the third one is used as a support for the coil trap and cap to be imbed onto it.

On the top, the center component presents four holes that are used to screw it onto the cap and one larger central one that reaches through up to the magnet surface which is used to pour the silicone into the chamber.

When the silicone is set the core center would remain stuck into the mat as it is too complex to be removed without damaging the mat itself.
To solve this problem we decided to create the core center in a material that could be easily dissolved in water.
This material is called BVOH and is a water-soluble filament that can be used as a support material for 3D printing.
To speed up the dissolution process we decided to print the core center with low infill and shave some material off the smallest cylinder to allow the water to reach the BVOH more easily.

\subsubsection{Distance magnet-coil}
The middle cylinder is what dictates the distance between the magnet and the coil.
This distance is crucial as it will determine the strength of the magnetic field that will reach the magnet.
After multiple prototypes, we landed on a distance of 3.5mm between the magnet and the coil.
In theory, the distance could be lower but we had to take into account the flexing of the membrane due to the pressure on it generated by the finger grasping the device.

\subsubsection{Coil trap}
This component has two functions, the first one is to house the coil and its heatsink and the second one is to trap the coil mechanically inside the mat.
Our main design limitations were that we couldn't glue the trap to the mat and that the trap needed to be able to flex with the mat and coil.
\begin{figure}
    \centering
    \begin{subfigure}[b]{0.475\textwidth}
        \centering
        \resizebox{\textwidth}{!}{
            \includegraphics{Chapters/Chapter5/Flexible_Mat_Prototypes/Figures/coil_trap_expl.png}
        }
        \caption{Coil trap exploded view.}
        \label{fig: coil_trap_expl}
    \end{subfigure}
    \hfill
    \begin{subfigure}[b]{0.475\textwidth}
        \centering
        \resizebox{\textwidth}{!}{
            \includegraphics{Chapters/Chapter5/Flexible_Mat_Prototypes/Figures/coil_trap_closed.png}
        }
        \caption{Coil trap with coil closed view.}
        \label{fig: coil_trap_closed}
    \end{subfigure}
    \caption{Coil trap model.}
    \label{fig: mat_mold_core_trap}
\end{figure}
The design we came up its a thin square structure with a small higher border where the coil is positioned (the lower part in figure \ref{fig: mat_mold_core_trap}).
The coil is then covered by a thin square plate that is screwed to the trap (the higher part in figure \ref{fig: mat_mold_core_trap}).
On the side of this square, we have four thin fins that will remain inside the silicone structure of the mat, mechanically blocking the trap inside.
\begin{figure}
    \centering
    \resizebox{0.7\textwidth}{!}{
        \includegraphics{Chapters/Chapter5/Flexible_Mat_Prototypes/Figures/coil_trap_in_mat.png}
    }
    \caption{Coil trap placed inside the mat.}
    \label{fig: coil_trap_in_mat}
\end{figure}
As the coil trap is very thin and it's printed in TPU it can easily flex with the mat and the coil.

\subsubsection{Production method}
To create a new mat various steps need to be followed:
\begin{itemize}
    \item \textbf{Components printing: } All the components of the core and the cavity need to be printed.
    \item \textbf{Cavity assembling: } We first need to glue the magnet to the pedestal and then place it inside the hole at the bottom of the big part.
    \item \textbf{Core assembling: } We start by placing the bottom part of the coil trap on the smallest cylinder of the core center, then we screw the mold core cap and bridge to the core center and finally we screw the bridge to the cavity.
    \item \textbf{Silicone casting: } We mix the silicone and pour it first into the big hole on top of the cap with a syringe until the chamber is filled, we can notice when it's full by observing the holes at the bottom of the cavity and the membrane holes on the core center.
    We then pour the rest of the silicone into the cavity until the silicone covers the coil trap fins.
    \item \textbf{Removing the mat from the mold: } After the silicone is set we can remove the mat from the mold by unscrewing the bridge and pulling it out.
    Now we can also remove the pedestal and core cap.
    \item \textbf{Removing the core center: } We then place the mat in a container filled with water and let it dissolve the core center.
    \item \textbf{Finalizing the mat: } After the core center is dissolved we can place the coil with its heatsink inside the trap and screw the cover on.
\end{itemize}

% -- Subsection 3.3
\subsection{Design faults and problems}



\subsubsection{Membrane fragility}

\subsubsection{Overall system flexibility}

