\subsection{Design of the membrane}
\label{sec: Design_of_the_membrane}
The design goal for the membrane was to create a structure that could support the magnet just enough to win over its gravitational force so that any other force applied to it on the z-axis would be enough to move it and make it vibrate.
For the membrane design, we decided to use a simple Celtic-cross structure as we described previously in section \ref{sec: Membrane-magnet_system}.
This resulted in a membrane with a central cylindrical chamber used to trap the magnet which is suspended by four parallelepipedal arms.
\begin{figure}
    \centering
    \resizebox{0.9\textwidth}{!}{
        \includegraphics{Chapters/Chapter5/Flexible_Mat_Prototypes/Figures/membrane_v1.png}
    }
    \caption{Membrane virtual model of the small magnet prototype.}
    \label{fig: Membrane_v1_model}
\end{figure}

\subsubsection{Material stiffness and thickness}
While prototyping we experimented with the same two different silicone materials we described in section \ref{sec: Sleeve_production}.
We quickly realized that the softer silicone was too soggy as the membrane arms would need to be too thick to support the magnet.
We so decided to move forward with only the harder silicone which allowed us to create thinner arms.

\subsubsection{Membrane structure vs magnet dimensions}
The main prototypes we realized were designed with two different N52 cylindrical magnets in mind.
The first one is a small 10mm diameter and 2mm thick magnet, the second one is a 15mm diameter and 3mm thick magnet.
The two magnets also have very different weights, the small one weighs 1.13g while the big one 4.03g.
The first design of the membrane was based on the small magnet, thanks to it being lightweight it could be supported by a membrane with very thin arms (0.6mm) and a width of 4mm.
The membrane arms needed also to be long enough to allow the magnet to move freely on the z-axis, so we decided to make them 4mm long.
\begin{figure}
    \centering
    \resizebox{0.9\textwidth}{!}{
        \includegraphics{Chapters/Chapter5/Flexible_Mat_Prototypes/Figures/membrane_v1_section.png}
    }
    \caption{Membrane cross-section of the small magnet prototype (t->thickness, L->length).}
    \label{fig: Membrane_v1_section}
\end{figure}

Switching to the big magnet we realized that the membrane arms would need to be thicker to support the magnet, even if we made them wider (5mm).
This was due to the increased weight of the magnet and the increase in the arms' length (now 5mm) we needed to make to allow the magnet to move freely on the z-axis.
To find the minimal thickness we used the model described in subsection \ref{sec: Membrane_stiffness}.
We set a maximum deflection of 0.8mm for the membrane arms and calculated the thickness needed to support the magnet using equations \ref{eq: Beam_deflection} and \ref{eq: Beam_inertia}.
The results showed that the membrane arms would need to be at least 1.45mm thick to support the magnet, so we decided to make them 1.6mm thick to have a safety margin.

The next problem we encountered arose when we observed that the membrane arms were breaking at the connection with the cylindrical chamber.
This was the abrupt change of profile that was causing a stress concentration at that point.
To solve this problem we decided to add a small fillet to the connection between the arms and the chamber.
\begin{figure}
    \centering
    \resizebox{0.9\textwidth}{!}{
        \includegraphics{Chapters/Chapter5/Flexible_Mat_Prototypes/Figures/membrane_v2_section.png}
    }
    \caption{Membrane cross-section of the big magnet prototype.}
    \label{fig: Membrane_v2_section}
\end{figure}