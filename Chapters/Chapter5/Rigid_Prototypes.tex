\section{Rigid Prototypes}
The goal of this research is to develop a flexible device but before delving into flexible prototypes we started by designing some rigid prototypes. The rigid prototypes were designed to test the concept of the device and to understand the limitations of the technology.

% -- Subsection 2.1
\subsection{1st version - Dresda Coils testbed}
The first prototype was designed to test the capabilities of the Dresden coils.
In the previous research done by the HZDR team \cite{HZDR} they tested the coil using a simple piece of flexible magnetic tape as a membrane.

\subsubsection{Flexible magnetic membrane}
This membrane is shaped like a "fish" so the tail can be fixed on a plane and the head can be free to bend up and down.

\begin{figure}
    \centering
    \resizebox{0.2\textwidth}{!}{\includegraphics{Chapters/Chapter5/Rigid_Prototypes/Figures/Dresden_test.png}}
    \caption{Dresden coil HZDR test setup}
    \label{fig: Dresden_test}
\end{figure}

When the coil was powered, the magnetic field produced by the coil would repel the membrane and bend it up.
The coil would be powered with an AC signal at various frequencies, then one would need to keep his pulp suspended at a certain distance over the membrane and feel the vibration produced by the membrane.
The pulp needed to be suspended at a certain distance to avoid pressing on the membrane, this would have caused the membrane to stop vibrating.

\subsubsection{Adjustable height platform for coil and membrane}
The most important thing to solve was to find a way to keep the pulp at a certain distance from the membrane.
Firstly we designed a platform that could keep the finger steady.
\begin{figure}
    \centering
    \resizebox{0.5\textwidth}{!}{\includegraphics{Chapters/Chapter5/Rigid_Prototypes/Figures/finger_holder.png}}
    \caption{Finger platform}
    \label{fig: finger_platform}
\end{figure}

This platform was modeled to have an ergonomic cavity for the finger to rest in and a hole for the pulp to be suspended over the membrane.
The square hole is large enough to allow the "fish" membrane to move freely.
Under the hole, there is a large cavity where a mechanism is placed.
This mechanism is a platform where the coil and membrane can be placed in a configuration similar to the one used in the HZDR experiment.
The platform can then be raised or lowered to find the right distance between the membrane and the finger pulp.
\begin{figure}
    \centering
    \resizebox{0.5\textwidth}{!}{\includegraphics{Chapters/Chapter5/Rigid_Prototypes/Figures/adj_platform.png}}
    \caption{Adjustable platform}
    \label{fig: adj_platform}
\end{figure}
The mechanism is a simple screw that can be turned to raise or lower the platform.

\subsubsection{Prototype usability}
This prototype proved to be pretty finicky to use.
The main problem was that distance couldn't be easily adjusted as the platform wouldn't remain stable enough on the screw.
This meant that finding an optimal distance between the pulp and the membrane was difficult, especially because different people have different pulp thicknesses.
Even if a distance that was good for one person was found, the vibration produced by the membrane was very weak and could barely be felt.

% -- Subsection 2.2
\subsection{Wearable Rigid Prototypes}
With this prototype, we wanted to try fixing most of the problems encountered with the previous prototype.
Firstly we decided to substitute the Dresden coil with the Flexar one as it was more powerful.
Then we wanted to decouple the membrane from the coil to prevent the membrane from being pressed by the pulp and remove the need for an adjustable platform to keep the coil at the right distance.
Finally, we wanted to make the device wearable.

\subsubsection{Finger-Membrane interface}
After some testing, we found that a good way to decouple the membrane from the coil and better the transmission of vibrations to the finger was to use a small high-performance magnet attached directly to the finger pulp.

For our testing, we used an N42-grade neodymium cylindrical magnet with a diameter of 10mm and a height of 3mm.
This magnet was fixed to the index pulp of the tester using some non-toxic glue and then he would be able to feel substantial vibrations by moving his pulp closer to the powered-on coil (with an AC signal).

Considering this knowledge, we designed a silicon sleeve that could be worn on the finger, this sleeve has a cavity for the magnet to be inserted into and be kept near the skin.
\begin{figure}
    \centering
    \resizebox{1\textwidth}{!}{
        \begin{subfigure}[b]{0.45\textwidth}
            \includegraphics[width=\textwidth]{Chapters/Chapter5/Rigid_Prototypes/Figures/silicon_sleeve_front.png}
        \end{subfigure}
        \begin{subfigure}[b]{0.45\textwidth}
            \includegraphics[width=\textwidth]{Chapters/Chapter5/Rigid_Prototypes/Figures/silicon_sleeve_back.png}        
        \end{subfigure}
    }
    \caption{Finger silicon sleeve front and back view}
    \label{fig: finger_sleeve}
\end{figure}

This silicon sleeve was designed to be scalable for different finger sizes and to be easily worn and removed.
The part was produced by silicon casting inside a two part 3D printed mold.

The design is composed of three parts:
\begin{itemize}
    \item \textbf{Silicon sleeve: } this part was modeled by us on the profile of a real 3d scanned index finger, it can be automatically scaled by specifying the index's width as all the measurements are based on that value.
    \item \textbf{Magnet hole: } the hole is designed based on the diameter of the magnet and its height. We also had to find the right height tolerance between the magnet and the pulp to avoid that it could press on the finger too much, impeding vibrations.
    \item \textbf{Mounting wings: } on the sides of the sleeve we have two parallelepidical wings that are used to mount the sleeve to the structure were the coil will be attached (described in the following section). 
\end{itemize}

Another design problem to solve was the positioning of the magnet, as the design had the goal to be adaptable to differest finger tip sizes we had to consider fingers with widths varying from
For the magnet position we chose the center to be placed on the symmetry axis of the pulp, then we based the design on index fingers with widths between 13 and 20mm \cite{index_fingers_width}.
That meant that for 13mm fingers the magnet size was barely smaller than the finger's width, so knowing that the finger tends to get even narrower toward the tip, we had to place the magnet center closer to the first interphalangeal fold rather than to the pulp's center.

The part went through multiple iterations to land on the right thicknesses for the sleeve itself, it needed to be thin enough to be adaptable to multiple fingers (considering similar widths) but not thoo thin as to be durable enough. 

\subsubsection{Keep the distance from the coil}
We then focused on a structure able to keep the coil at a fixed distance from the slicon sleeve and magnet.
The design goals for this device were that it's should be lightweight, easily wearable and adaptable to multiple sleeves' sizes.
After multiple iterations we landed on a three parts design.
\begin{figure}
    \centering
    \resizebox{1\textwidth}{!}{
        \includegraphics{Chapters/Chapter5/Rigid_Prototypes/Figures/sleeve_holder.png}
    }
    \caption{Silicon finger sleeve holder}
    \label{fig: sleeve_holder}
\end{figure}
The first part is the structure were the sleeve can be attached to, this is done by inserting the mounting wings inside the two squared holes on the bottom of the part.

\subsubsection{Heat dissipation}

