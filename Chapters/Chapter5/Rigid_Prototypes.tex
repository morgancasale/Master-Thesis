\section{Rigid Prototypes}
In conducting this research, careful consideration was given to the selection of tools and technologies that would best support the implementation, experimentation, and analysis processes. This section provides an overview of the programming languages, machine learning frameworks, data processing tools, and other technologies employed throughout the research.

% -- Subsection 2.1
\subsection{1st version - Dresda Coils testbed}
The primary programming language for this research was Python, chosen for its versatility, rich ecosystem, and widespread use in machine learning. Python's readability and extensive libraries facilitated efficient coding and experimentation.

\subsubsection{Dresda Coils}
\subsubsection{Flexible magnetic membrane}
\subsubsection{Adjustable height platform for coil and membrane}

% -- Subsection 2.2
\subsection{Wearable Rigid Prototypes}

\subsubsection{Finger-Membrane interface}
PyTorch \cite{PyTorch} was selected as the primary deep learning framework due to its flexibility which aligned well with the nature of the tasks involving customized model architectures and strong community support which contributed to a fluid development experience.

PyTorch is an open-source machine learning framework developed by Facebook's AI Research lab (FAIR). Its design philosophy emphasizes flexibility, enabling researchers and practitioners to tailor datasets and training procedures to specific requirements. The flexibility offered by PyTorch manifests in two key areas: custom datasets and custom training/validation phases. PyTorch supports automatic differentiation, making it easier to implement and experiment with complex neural network architectures. Its community, extensive documentation, and seamless integration with hardware accelerators like GPUs contribute to its popularity in both research and production.

PyTorch facilitates the creation of the custom dataset through the \textit{torch.utils.data. Dataset} class.\\ 
By inheriting from this class and implementing the \textit{\_\_len\_\_} and \textit{\_\_getitem\_\_} methods, it's been possible to define datasets tailored to data structure and format used. This capability is invaluable when working with diverse data types, such as images or time-series, allowing seamless integration into PyTorch's data loading utilities.

PyTorch's flexibility extends to the training and validation phases, enabling users to define custom training loops, loss functions, and evaluation metrics. This is crucial for experimenting with novel architectures, incorporating domain-specific metrics, or implementing advanced training techniques. The ability to seamlessly integrate custom logic into the training process empowers researchers to push the boundaries of model development.

The deep understanding of this framework and its functionalities has been of primary importance for using third-party models like \textit{HRNet}\cite{sun2019deep} integrating with a customized dataset and ad hoc training and validation phase.

\subsubsection{Keep the distance from the coil}

\subsubsection{Heat dissipation}

%-----------------------------------
%	SUBSECTION 3
%-----------------------------------

%\subsection{Flexible Mat Prototypes}

%\subsubsection{Pandas}
Pandas is a powerful data manipulation and analysis library for Python. It provides data structures like DataFrames that facilitate the handling of structured data. Pandas excels in data cleaning, manipulation, and exploration tasks, offering a plurality of functions for indexing, merging, grouping, and aggregating data. Its integration with other libraries, such as NumPy, makes it a go-to choice for working with labeled data and time series.

Pandas DataFrames offer a convenient and versatile way to handle tabular data, making them an excellent choice for storing and preprocessing data before creating datasets in PyTorch. The integration between Pandas and PyTorch simplifies the transition from data exploration to model training.

%\subsubsection{NumPy}
NumPy is a fundamental library for numerical operations in Python. It provides support for large, multi-dimensional arrays and matrices, along with an assortment of high-level mathematical functions to operate on these arrays.

Moreover, NumPy arrays and PyTorch tensors share several similarities, making them interchangeable in many contexts. These similarities contribute to a smooth integration between the two libraries, facilitating data manipulation and interoperability.\\
The compatibility between NumPy and PyTorch simplifies data exchange and promotes a cohesive workflow in mixed-library environments.

%-----------------------------------
%	SUBSECTION 4
%-----------------------------------
%\subsection{Visualization Libraries}
%\subsubsection{Matplotlib and Seaborn}
Matplotlib is a versatile 2D plotting library for Python. Seaborn is a statistical data visualization library built on top of Matplotlib. It provides a high-level interface for creating attractive and informative statistical graphics. Seaborn simplifies the process of generating complex visualizations with concise syntax. It is particularly useful for exploring relationships in datasets through specialized plots for categorical data, distribution plots, and regression plots.\\
All the plots presented in the \textit{Evaluation of Training Dataset} (\textbf{\ref{Chapter5/EvalTraining}}) and in the \textit{Evaluation of Test Datasets} (\textbf{\ref{Chapter5/EvalTest}}) sections are created using this library.
