\subsection{Biology of Haptic Sensing}
The human tactile sensing system can \textbf{measure specific properties of materials}, such as
temperature, texture, shape, force, fine-form features, mass distribution, friction, hardness
and viscoelasticity, \textbf{through physical contact} between the human skin and the object.
Even the \textbf{changing state of the interaction}, such as gravitational and inertial effects, can be perceived through the sense of touch. 
As the sensing system works through the skin, it doesn't rely on a localized sensory organ but behaves as a \textbf{distributed system}, also \textbf{different parts of the body have different thresholds of sensitivity}.
For these reasons, it's difficult to treat a tactile signal as a well-defined quantity like visual and audio signals and its complex nature makes it \textbf{difficult to replicate} its functioning in science or engineering tasks.

The sense of touch is based on the somatosensory system, which is a \textbf{ complex system of nerve endings and touch receptors} in the skin. The somatosensory system is composed of four main types of receptors:
\begin{itemize}
    \item \textbf{Mechanoreceptors} - These are the most common type of tactile receptors in the skin. They are responsible for sensing \textbf{pressure}, \textbf{vibration}, \textbf{stretching}, and \textbf{brushing}.
    \item \textbf{Thermoreceptors} - These receptors are responsible for \textbf{sensing temperature} changes in the skin. There are two main types of thermoreceptors: warm receptors and cold receptors.
    \item \textbf{Nociceptors} - These receptors are responsible for \textbf{sensing pain and tissue damage}. They are activated by noxious stimuli, such as extreme temperatures, pressure, or chemicals.
    \item \textbf{Proprioceptors} - These receptors are responsible for \textbf{sensing the position and movement} of the body. They are located in the muscles, tendons, and joints, and provide feedback to the brain about the relative position between different parts of the body.
\end{itemize}

The \textbf{most important receptors for haptic feedback are the mechanoreceptors}, they react to mechanical stimuli by producing \textbf{signals in the form of streams of voltage pulses} at high frequencies, the stronger the stimuli higher the frequency of the pulses. When the cell adapts to the stimulus, the pulse frequency subsides to its normal rate.
Considering the goal of this research we can \textbf{focus} on the \textbf{mechanoreceptors} that are responsible for \textbf{sensing pressure and vibration}, these are the \textbf{Pacinian corpuscles} and the \textbf{Meissner corpuscles}.
The first ones are more sensible to \textbf{high-frequency} vibrations (200-550Hz), while the second ones are more for \textbf{low-frequency vibrations} (20-40Hz) \cite{Alg_Wearable_Tech_Nicole}.