\section{State of the Art}
For now, state of the art in haptic feedback is still in its infancy.
The main reason for this is the complexity of the human sense of touch which we still don't understand fully.
The best technology we have for now to reproduce haptic feedback is the piezoelectric actuator.

\subsection{Piezoelectric actuators}
Piezoelectric actuators are a very interesting technology based on the piezoelectric effect.
Materials exhibiting this effect, such as certain ceramics and crystals, possess the ability to convert electrical energy into mechanical motion, and vice versa.

The operating principle of piezoelectric actuators relies on the application of an electric field across the piezoelectric material. This electric field induces a deformation within the material, causing it to expand or contract depending on the polarity of the applied voltage. This minute deformation translates into highly precise mechanical displacement, enabling piezoelectric actuators to achieve nanometer-scale resolutions with remarkable speed and accuracy.

One of the defining characteristics of piezoelectric actuators is their rapid response time. Unlike traditional electromagnetic actuators, which may suffer from inertia and mechanical backlash, piezoelectric actuators can swiftly change their state in response to electrical signals.


\subsubsection{Force performances}

\subsubsection{Frequency response}

\subsubsection{WeArt implementation}
