\section{Introduction to Haptic Feedback}
\label{Introduction_Haptics}
Computer Haptic Feedback or Haptics, in short, is the research field that deals with the need to be able to digitalize the human sense of touch and reproduce it. Despite the research done in this field since the mid-20\textsuperscript{th} century, the technology is still in its infancy. The main reason for this is the complexity of the human sense of touch which we still don't understand fully.
This, in turn, doesn't allow us to even approximately match the capabilities of the human sense of touch; but we can still use this infant technology to reproduce simple sensations.
This sensation can be used in many fields, from the entertainment industry to the medical field, from the military to the automotive industry to convey information that we do not normally acquire via touch, such as notifications
and warnings related to particular events, guidance instructions and even crude reproduction of textures.
In this chapter, we will give an overview of the human sense of touch and the state of art technologies used to reproduce it.

% -- Subsection 1
\subsection{Biology of Haptic Sensing}
The human tactile sensing system can measure specific properties of materials, such as
temperature, texture, shape, force, fine-form features, mass distribution, friction, hardness
and viscoelasticity, through physical contact between the human skin and the object.
Even the changing state of the interaction, such as gravitational and inertial effects, can
be perceived through the sense of touch. 
As the sensing system works through the skin, it doesn't rely on a localized sensory organ but behaves as a distributed system, also different parts of the body have different thresholds of sensitivity.
For these reasons, it's difficult to treat a tactile signal as a well-defined quantity like visual and
audio signals and its complex nature makes it difficult to replicate its functioning in
science or engineering tasks.

The sense of touch is based on the somatosensory system, which is a complex system of nerve endings and touch receptors in the skin. The somatosensory system is composed of four main types of receptors:
\begin{itemize}
    \item \textbf{Mechanoreceptors} - These are the most common type of tactile receptors in the skin. They are responsible for sensing pressure, vibration, stretching and brushing.
    \item \textbf{Thermoreceptors} - These receptors are responsible for sensing temperature changes in the skin. There are two main types of thermoreceptors: warm receptors and cold receptors.
    \item \textbf{Nociceptors} - These receptors are responsible for sensing pain and tissue damage. They are activated by noxious stimuli, such as extreme temperatures, pressure, or chemicals.
    \item \textbf{Proprioceptors} - These receptors are responsible for sensing the position and movement of the body. They are located in the muscles, tendons, and joints, and provide feedback to the brain about the relative position between different parts of the body.
\end{itemize}

The most important receptors for haptic feedback are the mechanoreceptors, they react to mechanical stimuli by producing signals in the form of streams of voltage pulses at high frequencies, the stronger the stimuli higher the frequency of the pulses. When the cell adapts to the stimulus, the pulse frequency subsides to its normal rate.
Considering the goal of this research we can focus on the mechanoreceptors that are responsible for sensing pressure and vibration, these are the Pacinian corpuscles and the Meissner corpuscles.
The first ones are more sensible to high-frequency vibrations (200-550Hz), while the second ones are more for low-frequency vibrations (20-40Hz). \cite{Alg_Wearable_Tech_Nicole}


% -- Subsection 2
\subsection{Vibration Propagation in the finger's pulp}

% -- Subsection 3
\input{Chapters/Chapter3/Introduction_Haptics/Strength_of_Sensation.tex}