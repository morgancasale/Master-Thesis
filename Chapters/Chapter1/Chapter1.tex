\chapter{Introduction} % Main chapter title

\label{Chapter1} % Change X to a consecutive number; for referencing this chapter elsewhere, use \ref{ChapterX}

%----------------------------------------------------------------------------------------
%	SECTION 1
%----------------------------------------------------------------------------------------
\section{Thesis objective}

The world of haptics is an exciting field in the world of robotics, it was born from the will to bridge even further the digital world with the real one.
After digitalizing the sense of sight and hearing, the sense of touch is the next logical frontier to conquer.
The sense of touch is a very important sense for humans, it is the first sense that develops in the womb and it is the first sense that is used by a newborn to explore the world.
It is also the sense that is the most difficult to replicate in a virtual environment.

In past years there has been a lot of research in the field of haptics, but the field is still in its infancy.
Even if the current best haptic devices are able to provide vibration, force and temperature cues to the user, they are still far from being able to replicate the sense of touch realistically.

The most important components of haptics are vibrations and forces, they are the most common cues that are used to convey information to the user.
The generation of these types of cues is usually handled by piezoelectric actuators as they can generate vibrations with a high bandwidth (especially in the high range) and precision exerting notable force.

The only drawbacks of piezoelectric actuators are that they are rigid components, they cannot be bent or stretched, and they are not able to generate forces in the low range.

The objective of this thesis is to study a different type of actuator, voice coils based on flexible coils, and to compare them with piezoelectric actuators.
The main advantage of voice coils is that they can generate vibration with enough force also at low frequencies. 
In this research, we want to understand if we can create a haptic device that can be stretchable and possibly wearable by using flexible coils.

We will study the strengths and weaknesses of such type of actuator and propose multiple designs for a haptic device created with this technology.

%----------------------------------------------------------------------------------------
%	SECTION 2
%----------------------------------------------------------------------------------------
\section{Necessary background (?)}
\label{Chapter1/Dataset}
The algorithm is designed around \textit{TASI EROSS IOD Simulated Dataset N°2}, which consists of grayscale images of the satellite; see figure \textbf{\ref{fig:Samples}}.\\
The training dataset is composed of sixteen trajectories each of 900 images. Each trajectory covers the distance range from 200 to 20 cm to the target and the difference between each subsequent frame captured by each camera is 0.2 cm. Each image is of size 512x512 pixels and it's paired with ground truth 6DOF poses (position and orientation).
\begin{figure}[th]
    \centering
    \includegraphics[scale=0.48]{Figures/Chapter1/DatasetExample.png}
    \caption[Sample images from \textit{TASI EROSS IOD Simulated Dataset N°2}]{Sample images from the \textit{TASI EROSS IOD Simulated Dataset N°2}}
    \label{fig:Samples}
\end{figure}
\newpage
The data acquisition has been performed as follow:
\begin{itemize}
    \item \textbf{Non Prepared scenario:} LAR view.
    \item \textbf{Natural Illumination:} Full illumination (sun @45°, 130k lux).
    \item \textbf{Illumination system:} ON (150lm x6 LEDs).
    \item \textbf{Simulated Camera Settings:}
        \begin{itemize}
            \item Shutter speed: 20
            \item ISO: 5
            \item Aperture: 4
            \item FOV: 67.8°
        \end{itemize}
    \item \textbf{Reference System:}
        \begin{itemize}
            \item Left-handed XYZ reference.
            \item Origin [0,0,0]:
            \begin{itemize}
                \item XY: zeroes on the vertical symmetry axis of the Target.
                \item Z: Positive towards contact, zeroed on the lowest contact point of the LAR.
            \end{itemize}
            \item All units are in centimeters (cm).
        \end{itemize}
    \item \textbf{Trajectories:}
        \begin{itemize}
            \item all trajectories follow the same XYZ coordinates.
            \item the rotation is considered with Euler angles as Pitch (around axis x), Yaw (around axis y) and Roll (around axis z), all positive counterclockwise. Trajectories' specifics are reported in table \textbf{\ref{tab:trajectories}}.
            \item Camera pointing XY in [-46, +20].
        \end{itemize}
\end{itemize}

\begin{figure}[th]
    \centering
    \includegraphics[scale=0.8]{Figures/Chapter1/UPS.png}
    \caption[Model's UPS]{Model's UPS}
    \label{fig:UPS}
\end{figure}

\begin{table}[H]
\caption{Specifics of the simulated training trajectories.}
\label{tab:trajectories}
\centering
\begin{tabular}{l | l l l}
\toprule
Trajectory & Roll(°) & Pitch(°) & Yaw(°)\\
\midrule
TRAY\_1 & 0 & 0 & 0\\
TRAY\_2 & 0 & 0 & -1\\
TRAY\_3 & 0 & 0 & -2\\
TRAY\_4 & 0 & 0 & -3\\
TRAY\_5 & 0 & 0 & -4\\
TRAY\_6 & 0 & 0 & -5\\
TRAY\_7 & 0 & 1 & 0\\
TRAY\_8 & 0 & 2 & 0\\
TRAY\_9 & 0 & 3 & 0\\
TRAY\_10 & 0 & 4 & 0\\
TRAY\_11 & 0 & 5 & 0\\
TRAY\_12 & -1 & 0 & 0\\
TRAY\_13 & -2 & 0 & 0\\
TRAY\_14 & -3 & 0 & 0\\
TRAY\_15 & -4 & 0 & 0\\
TRAY\_16 & -5 & 0 & 0\\
\bottomrule
\end{tabular}
\end{table}

The \textit{TASI EROSS IOD Simulated Dataset N°3} along with the \textit{Less\_Difficult\_Trajectory} and \textit{Difficult\_Trajectory} are used as the test dataset. The \textit{TASI EROSS N°3} is composed of two trajectories, each of which was captured with two different camera positions. Each trajectory has 900 images.\\
The \textit{TRAY\_A} starts with +15° on Yaw and linearly converge toward 0 on contact, while \textit{TRAY\_B}, \textit{Less\_Difficult\ Trajectory} and \textit{Difficult\_Trajectory} present multiple errors on RPY converging toward 0 on contact as well.


%----------------------------------------------------------------------------------------
%	SECTION 3
%----------------------------------------------------------------------------------------
\section{Thesis structure}
The thesis is structured in further five chapters:
\begin{description}
    \item[Chapter 2] - \textit{Background}:\\ This chapter provides a comprehensive overview of key concepts necessary for the correct understanding of this work, with a focus on monocular camera models, perspective projection, pose estimation, and a general introduction to deep learning models.
    \item[Chapter 3] - \textit{State-of-art}:\\ This chapter delves into monocular pose estimation methods, covering classic approaches like RANSAC and SfM, and exploring modern techniques such as end-to-end learning with networks like PoseNet and Mask R-CNN. The chapter also introduces feature learning, emphasizing CNN-based methods like HRNet for predicting 2D landmark locations. Moreover, some studies about spacecraft pose estimation and their use of deep learning architectures are presented. The chapter also delves into point set alignments, highlighting the widely used and advanced algorithms like Coherent Point Drift (CPD) technique employed in the method for final pose estimation.
    \item[Chapter 4] - \textit{Algorithms and Methods}: \\
    This chapter delves into the methodology's core algorithms and techniques. It outlines the offline architecture, detailing the 2D-3D correspondence process, landmark regression, and the neural network-based landmark mapping. The chapter then presents the online architecture, covering real-time processing and the Coherent Point Drift technique for pose estimation. Implementation challenges and dataset considerations are also discussed, providing a comprehensive overview of the applied methods.
    \item[Chapter 5] - \textit{Implementation and Experiments}:\\
    This chapter presents the tools and technologies employed for the project implementation and the evaluation metrics for pose estimation, Landmark Regression, Landmark Mapping are described. The chapter culminates in the assessment of both training and test datasets, showcasing the method's robustness and generalization across diverse scenarios. Overall, it provides comprehensive exploration of the research's implementation and experimentation phases.
    \item[Chapter 6] - \textit{Discussions and Conclusions}:\\
    The Chapter delves into challenges faced by on-board AI systems in space missions, focusing on verifiability and computational load. It emphasizes the significance of minimizing translation errors for accurate maneuvering in the proposed multi-model configuration. The section explores potential improvements, including enhanced landmark selection and strategies to fortify system robustness.
\end{description}
