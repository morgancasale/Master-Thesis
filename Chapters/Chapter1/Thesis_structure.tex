\section{Thesis structure}
The thesis is structured in further five chapters:
\begin{description}
    \item[Chapter 2] - \textit{Background}:\\ This chapter provides a comprehensive overview of key concepts necessary for the correct understanding of this work, with a focus on monocular camera models, perspective projection, pose estimation, and a general introduction to deep learning models.
    \item[Chapter 3] - \textit{State-of-art}:\\ This chapter delves into monocular pose estimation methods, covering classic approaches like RANSAC and SfM, and exploring modern techniques such as end-to-end learning with networks like PoseNet and Mask R-CNN. The chapter also introduces feature learning, emphasizing CNN-based methods like HRNet for predicting 2D landmark locations. Moreover, some studies about spacecraft pose estimation and their use of deep learning architectures are presented. The chapter also delves into point set alignments, highlighting the widely used and advanced algorithms like Coherent Point Drift (CPD) technique employed in the method for final pose estimation.
    \item[Chapter 4] - \textit{Algorithms and Methods}: \\
    This chapter delves into the methodology's core algorithms and techniques. It outlines the offline architecture, detailing the 2D-3D correspondence process, landmark regression, and the neural network-based landmark mapping. The chapter then presents the online architecture, covering real-time processing and the Coherent Point Drift technique for pose estimation. Implementation challenges and dataset considerations are also discussed, providing a comprehensive overview of the applied methods.
    \item[Chapter 5] - \textit{Implementation and Experiments}:\\
    This chapter presents the tools and technologies employed for the project implementation and the evaluation metrics for pose estimation, Landmark Regression, Landmark Mapping are described. The chapter culminates in the assessment of both training and test datasets, showcasing the method's robustness and generalization across diverse scenarios. Overall, it provides comprehensive exploration of the research's implementation and experimentation phases.
    \item[Chapter 6] - \textit{Discussions and Conclusions}:\\
    The Chapter delves into challenges faced by on-board AI systems in space missions, focusing on verifiability and computational load. It emphasizes the significance of minimizing translation errors for accurate maneuvering in the proposed multi-model configuration. The section explores potential improvements, including enhanced landmark selection and strategies to fortify system robustness.
\end{description}