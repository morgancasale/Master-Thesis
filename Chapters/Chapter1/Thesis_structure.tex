\section{Thesis structure}
The thesis is structured in further five chapters:
\begin{description}
    \item[Chapter 2] - \textit{Background}:\\ This chapter provides a comprehensive overview of all physical laws and phenomena that need to be known to describe the working principle of a voice actuator style haptic actuator based on flexible PCB technology. The chapter covers the basics of the electromagnetism laws that will be used to describe the magnetic field generation of the coil and its interaction with a magnet to produce mechanical force.
    We will then talk about Flexible PBC coils physical characteristics, strengths and flaws.

    \item[Chapter 3] - \textit{Overview of Haptic Feedback}:\\ This chapter provides an overview of haptic feedback talking about how the haptic human perception works and its limits. We will then talk about the leading technology used in haptic feedback devices and its strengths and weaknesses.

    \item[Chapter 4] - \textit{Powering circuit design}: \\
    This chapter explains the design procedure for the entire control system for the haptic actuator. We will start by describing the controller characteristics and used hardware, then we will talk about a possible design for the power amplifying circuit.

    \item[Chapter 5] - \textit{Implementation and Prototypes}:\\
    This chapter presents all the prototypes we have built and tested during the development of the haptic actuator. Starting from a simple testbed for planar coils, we will then talk about a more advanced wearable prototype to finally pass to a flexible prototype. For each prototype, we will describe the design choices, the building process, their strengths and issues.


    \item[Chapter 6] - \textit{Discussions and Conclusions}:\\
    In this last chapter, we will discuss the results obtained from the prototypes and the possible future developments of the technology. We will also talk about the other applications of PCB coils.
\end{description}