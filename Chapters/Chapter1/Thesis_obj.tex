\section{Thesis objective}

The world of haptics is an exciting field in the world of robotics, it was born from the will to bridge even further the digital world with the real one.
After digitalizing the sense of sight and hearing, the sense of touch is the next logical frontier to conquer.
The sense of touch is a very important sense for humans, it is the first sense that develops in the womb and it is the first sense that is used by a newborn to explore the world.
It is also the sense that is the most difficult to replicate in a virtual environment.

In past years there has been a lot of research in the field of haptics, but the field is still in its infancy.
Even if the current best haptic devices are able to reproduce vibration, force and temperature cues to the user, they are still far from being able to replicate the sense of touch realistically.

The most important components of haptics are vibrations and forces, they are the most common cues that are used to convey information to the user.
The generation of these types of cues is usually handled by piezoelectric actuators as they can generate vibrations with a high bandwidth (especially in the high range) and precision exerting notable force.

The only drawbacks of piezoelectric actuators are that they are rigid components, they cannot be bent or stretched, and they are not able to generate forces in the low range.

The objective of this thesis is to study a different type of actuator, voice coils based on flexible coils, and to compare them with piezoelectric actuators.
The main advantage of voice coils is that they can generate vibration with enough force also at low frequencies. 
In this research, we want to understand if we can create a haptic device that can be stretchable and possibly wearable by using flexible coils.

We will study the strengths and weaknesses of such type of actuator and propose multiple designs for a haptic device created with this technology.