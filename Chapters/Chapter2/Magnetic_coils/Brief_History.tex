\subsection{Brief History}

The \textbf{connection between electricity and magnetism} was first demonstrated by \textbf{Hans Christian Oersted} in 1820 when he observed that an electric current flowing through a wire could deflect a nearby magnetic needle.

Meanwhile, the creation of the \textbf{first practical electromagnet} is credited to Wi\textbf{lliam Sturgeon} and \textbf{André-Marie Ampère} who, after Oersted's discovery, experimented with creating coil windings wrapped around an iron core which allowed them to achieve much stronger magnetic fields.

During the 1830's \textbf{Michael Faraday}'s discovery of electromagnetic induction further advanced the understanding of magnetic fields and coils. Faraday demonstrated that a \textbf{changing magnetic field could induce an electric current in a nearby conductor}, laying the groundwork for transformers and modern electrical generators.

The latter half of the 19th century saw rapid advancements in electrical engineering. Innovations like early \textbf{electric generators} (dynamos), \textbf{transformers}, and \textbf{electric motors} heavily \textbf{relied on magnetic coils} for their operation. Researchers such as \textbf{Nikola Tesla} and \textbf{Thomas Edison} further developed these technologies.

Magnetic coils continue to play a vital role in various fields, including \textbf{power generation}, \textbf{telecommunications}, \textbf{electronics}, and \textbf{medical imaging} (such as MRI machines). With advancements in materials science and manufacturing techniques, magnetic coils have become more efficient, compact, and versatile.

In recent years, as the use of \textbf{PCBs} has become widespread, researchers started experimenting with \textbf{creating coil windings utilizing this technology}.