\subsection{Brief History}

The connection between electricity and magnetism was first demonstrated by Hans Christian Oersted in 1820 when he observed that an electric current flowing through a wire could deflect a nearby magnetic needle.

Meanwhile, the creation of the first practical electromagnet is credited to William Sturgeon and André-Marie Ampère who after Oersted's discovery experimented with creating coil windings wrapped around an iron core which allowed them to achieve much stronger magnetic fields.

During the 1830's Michael Faraday's discovery of electromagnetic induction further advanced the understanding of magnetic fields and coils. Faraday demonstrated that a changing magnetic field could induce an electric current in a nearby conductor, laying the groundwork for transformers and modern electrical generators.

The latter half of the 19th century saw rapid advancements in electrical engineering. Innovations like early electric generators (dynamos), transformers, and electric motors heavily relied on magnetic coils for their operation. Researchers such as Nikola Tesla and Thomas Edison further developed these technologies.

Magnetic coils continue to play a vital role in various fields, including power generation, telecommunications, electronics, and medical imaging (such as MRI machines). With advancements in materials science and manufacturing techniques, magnetic coils have become more efficient, compact, and versatile.

In recent years, as the use of PCBs has become widespread, researchers started experimenting with creating coil windings utilizing this technology.