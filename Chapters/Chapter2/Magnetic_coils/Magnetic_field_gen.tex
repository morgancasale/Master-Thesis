\subsection{Magnetic field generation}
The strength of the magnetic field on the z-axis of the coil is derived from the Biot-Savart Law and is given by the formula
\begin{figure}
  \centering
  \resizebox{.7\linewidth}{!}{\includesvg[angle = 90, draft = false, width = 1\textwidth]{Chapters/Chapter2/Magnetic_coils/Figures/Solenoid&fluxOfB.svg}}
  \caption{Magnetic field generated by a solenoid.}
  \label{fig: Coil_magnetic_field}
\end{figure}


\begin{equation}
  B_z=\frac{N\mu Ir^2}{2(r^2+z^2)^\frac{3}{2}} \label{eq: Coil_magn_field}
\end{equation}


Where:
\begin{itemize}
  \item \( B_z \) is the magnetic field on the z-axis [T].
  \item \( \mu \) is the magnetic permeability of the medium [H/m].
  \item \( I \) is the current flowing through the wire [A].
  \item \( r \) is the radius of the coil [m].
  \item \( N \) is the number of turns of wire in the coil.
  \item \( z \) is the z-distance from the center of the coil [m].
\end{itemize}

If the coil lacks a core the permeability of free space is used instead of the core's permeability; instead if wound on a ferromagnetic core the permeability of the core is calculated as
\[\mu = \mu_0 \cdot \mu_r\]
where \( \mu_r \) is the relative permeability of the core material.

With the right material for the core, the magnetic field intensity can be highly increased compared to the field generated by the coil alone.

% TODO: IL FOTTUTO GRAFICO A BARRE DELLA PERMEABILITA'

\subsubsection{Magnetic Flux and Field relation}
We can also relate the magnetic field to the magnetic flux generated by the coil. The magnetic flux is given by the formula

\begin{equation}
  \Phi_B=B \cdot A \label{eq: Magnetic_flux_&_field}
\end{equation}

Where:
\begin{itemize}
  \item \( \Phi_B \) is the magnetic flux [Wb].
  \item \( B \) is the magnetic field [T].
  \item \( A \) is the area of the coil [m\(^2\)].
\end{itemize}