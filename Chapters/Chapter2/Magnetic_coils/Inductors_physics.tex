\subsection{Physics of inductors}
We will start this thesis by analyzing all the physics laws that govern the behavior of induction coils. This will allow us to understand their behavior with magnetic field generation and the power cost for their operation. 
\subsubsection{Inductance}
All \textbf{conductors have some inductance}, which may have either desirable or detrimental effects in practical electrical devices. The inductance of a circuit depends on the \textbf{geometry of the current path} and the \textbf{magnetic permeability} of \textbf{nearby materials}.

\begin{samepage}
    The \textbf{inductance} of a circuit depends on the \textbf{magnetic flux} and \textbf{current} flowing through it:
    
    \nopagebreak

    \begin{equation*}
        L = \frac{\Phi(i)}{i} \label{eq: Inductance_&_flux}
    \end{equation*}

    \nopagebreak

    Where:
    \begin{itemize}
        \item $L$ is the inductance [H].
        \item $i$ is the current [A].
        \item $\Phi(i)$ is the magnetic flux through the circuit [Wb].
    \end{itemize}
\end{samepage}    

\subsubsection{Reactance}
When a current signal is applied to an inductor, a flux is generated.
Then, considering Faraday's law of induction, any \textbf{change in flux} through a circuit induces an \textbf{electromotive force} ${\mathcal {E}}$, proportional to the rate of change of flux:

\begin{equation*}
    {\mathcal {E}} = -L \frac{d\Phi(t)}{dt}
\end{equation*}

We also know that by Lenz's law, the voltage across the inductor can be calculated as:

\begin{equation*}
    V = -L \frac{di}{dt}
\end{equation*}

Inductors resist changes in current due to the magnetic field they generate when current passes through them. When we apply a \textbf{sinusoidal signal} to our inductor, the current will be continuously \textbf{changing direction}. The \textbf{inductor's opposition} to these changes is represented as \textbf{reactance}.

Inductive reactance (\(X_L\)) is measured in ohms and is calculated using the formula:

\begin{equation*}
    X_L = 2\pi fL
\end{equation*}

Where:
\begin{itemize}
    \item \( X_L \) is the inductive reactance [\(\Omega\)]
    \item \( f \) is the frequency of the AC current [Hz]
    \item \( L \) is the inductance of the inductor [H]
\end{itemize}

\begin{samepage}
    We can then calculate the total impedance of the inductor as
    \nopagebreak

    \begin{equation*}
        Z = \sqrt{R^2 + X_L^2}
    \end{equation*}
    \nopagebreak

    Where:
    \nopagebreak

    \begin{itemize}
        \item \( Z \) is the total impedance [\(\Omega\)]
        \item \( R \) is the resistance of the inductor [\(\Omega\)]
        \item \( X_L \) is the inductive reactance [\(\Omega\)]
    \end{itemize} 
\end{samepage}


\subsubsection{Joule heating}
Any passive component in an electrical circuit will dissipate power in the form of heat. This is known as \textbf{Joule heating} and is caused by the resistance of the component.

\begin{samepage}
    The power dissipated in an inductor is given by the relation
    \nopagebreak

    \begin{equation}
        P = |I_{RMS}|^2R = \frac{|V_{RMS}|^2}{|Z|^2}R
        \label{eq: Joule_heating}
    \end{equation}
    \nopagebreak

    Where:
    \begin{itemize}
        \item \( P \) is the power dissipated in the inductor [W]
        \item \( I \) is the current flowing through the inductor [A]
        \item \( R \) is the resistance of the inductor [\(\Omega\)]
        \item \( V \) is the voltage across the inductor [V]
        \item \( Z \) is the total impedance of the inductor [\(\Omega\)]
    \end{itemize}
\end{samepage}

\subsubsection{Definition of Root Mean Square (RMS) values}
As we have seen in the previous paragraphs, the \textbf{power dissipated} by the coil depends on the \textbf{root mean square values} of the \textbf{current} and \textbf{voltage}.
We use the \textbf{RMS values} because they \textbf{allow us to compare} the power dissipated by the coil when powered in \textbf{AC} and \textbf{DC} conditions.

\begin{samepage}
    For DC signals these values are equal to the DC one, while for sinusoidal signals $V_{RMS}$ can be calculated as:
    \begin{equation*}
        V_{RMS} = \sqrt{\frac{1}{T} {\int_{T}^{0} {[f(t)]}^2\, {\rm d}t}}
    \end{equation*}
    \nopagebreak

    Where:
    \begin{itemize}
        \item \(T\) is the period of the input signal
        \item \(f(t)\) is the function of the signal 
    \end{itemize}
    \hfill
\end{samepage}

Then in case we're dealing with AC signals having a DC offset we can use the formula
\begin{equation*}
    V_{RMS_{AC+DC}} = \sqrt{V_{DC}^2 + V_{RMS_{AC}}^2}
\end{equation*} 
    
\begin{samepage}
    Some formulas for important waveforms:
    \nopagebreak

    \begin{figure}[H]
        \centering
        \resizebox{.9\linewidth}{!}{
            \begin{tabular}{|l l l|} % <-- Alignments: 1st column left, 2nd column left
    \hline
    \rowcolor{black} {\color{white} Name} & \quad {\color{white} Waveform} & \quad {\color{white} $V_{RMS}$} \cr
    \hline
    DC & \quad $V_P$ & \quad $V_P$ \cr
    \hline
    Sine Wave $[-V_P, V_P]$ & \quad $V_P \sin(2 \pi f t)$ & \quad $\frac{V_P}{\sqrt{2}}$ \cr
    \hline
    Polarized Sine Wave $[0, V_P]$ & \quad $\frac{V_P}{2} (\sin(2 \pi f t)+1)$ & \quad $\frac{V_P}{2} \sqrt\frac{3}{2}$ \cr
    \hline
    Square Wave $[-V_P, V_P]$ & \quad $V_P \text{sgn}(\sin(2 \pi f t))$ & \quad $V_P$ \cr
    \hline
    DC-shifted Square Wave $V_{DC}+[-V_P, V_P]$ & \quad $V_{DC} + V_P \text{sgn}(\sin(2 \pi f t))$ & \quad $\sqrt{V_{DC}^2 + V_P^2}$ \cr
    \hline
\end{tabular} % TODO: Fixare sta merda

        }
        \caption{RMS values for different waveforms.}
        \label{fig:RMS_table}
    \end{figure}
\end{samepage}
