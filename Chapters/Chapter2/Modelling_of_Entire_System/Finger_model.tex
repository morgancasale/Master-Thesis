\subsection{Finger grasping model}
\begin{samepage}
    At last, we have also to model the \textbf{finger grasping} the device, we can derive the model from the one used in the paper \cite{Finger_grasping_model} for the human finger.
    \nopagebreak

    \begin{figure}[H]
        \centering
        \includegraphics[width = 0.4\linewidth]{Chapters/Chapter2/Modelling_of_Entire_System/Figures/Model-of-two-soft-fingers-grasping-the-object.png}
        \caption{Model of two soft fingers grasping the object.}
        \label{fig: Finger_grasping_model}
    \end{figure}
\end{samepage}

This model describes two fingers grasping an object, for our case, we can simplify it to a \textbf{single finger grasping} the device.
Also, we can \textbf{neglect the friction} between the finger and the device as the device will be tested positioned on a flat surface with only the finger touching it from above.
\begin{figure}[H]
    \centering
    \resizebox{.7\linewidth}{!}{
        \input{Chapters/Chapter2/Modelling_of_Entire_System/Figures/finger_grasping_bond-graph.tex}
    }
    \caption{Bond graph of the finger grasping model.}
    \label{fig: Finger_grasping_bond_graph}
\end{figure}
