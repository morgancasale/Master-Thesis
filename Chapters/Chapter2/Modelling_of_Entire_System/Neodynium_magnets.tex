\subsection{Neodynium magnets (magnetic strength wrt class and dimensions)}
Neodymium magnets are a type of rare-earth magnet, they are the strongest type of permanent magnets made commercially.
They are made of an alloy of neodymium, iron, and boron and their strength depends on the percentage of neodymium in the alloy and on its crystalline structure.
They are classified based on their maximum energy product, which is the maximum amount of energy that can be stored in a magnet. Modern neodymium magnets start from N35 and go up to N52 (even N55), the higher the number the stronger their magnetic field.

Considering a cylindrical magnet with a radius $R_M$ and a thickness $t$ we can calculate the magnetic field generated by it at a distance $z$ from a pole surface using the formula \cite{Magnetic_field_perm_magnet}:
\begin{equation}
    B_M(z) = \frac{B_r}{2} \left( \frac{t+z}{\sqrt{R_M^2+(z+t)^2}} - \frac{z}{\sqrt{R_M^2+z^2}} \right) \label{eq:Magnetic_field_perm_magnet}
\end{equation} 

Where: 
\begin{itemize}
    \item $B_r$ is the remanence of the magnet [T]
    \item $R_M$ is the radius of the magnet [m]
    \item $t$ is the thickness of the magnet [m]
    \item $z$ is the distance from a pole surface of the magnet [m]
\end{itemize}

The remanence of a magnet is the magnetic field that remains in the magnet after the external magnetic field is removed and depends on the N grade of the magnet.

\begin{table}
    \centering
    \resizebox{.6\linewidth}{!}{\begin{tabular}{|l | l l|} % <-- Alignments: 1st column left, 2nd column left
    \hline
    \rowcolor{black} {\color{white} Goudsmit Grade} & \multicolumn{2}{|c|}{\color{white} Remanence $B_r\, [mT]$} \cr
    \hline
     & \quad min value & \quad typical value \cr
    \hline
    N35 & \quad 1170 & \quad 1210 \cr
    \hline
    N38 & \quad 1220 & \quad 1260 \cr
    \hline
    N40 & \quad 1260 & \quad 1290 \cr
    \hline
    N42 & \quad 1290 & \quad 1320 \cr
    \hline
    N45 & \quad 1320 & \quad 1370 \cr
    \hline
    N48 & \quad 1370 & \quad 1420 \cr
    \hline
    N50 & \quad 1400 & \quad 1460 \cr
    \hline
    N52 & \quad 1420 & \quad 1470 \cr
    \hline
\end{tabular} % TODO: Fixare sta merda
}
    \caption{Magnetic field remanence of different N grade neodymium magnets.}
    \label{tab: magnet_grades}
\end{table}