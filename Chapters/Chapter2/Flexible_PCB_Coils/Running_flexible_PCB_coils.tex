\subsection{Running flexible PCB coils}

\subsubsection{High current needs}
As discussed in the previous paragraph flexible coils have very high resistance so to produce even low magnetic fields \textbf{high current} must be provided.
Considering the power limit of the Flexar coil $P_{max} = 0.8W$ we can calculate the maximum current that can be provided to the coil with \ref{eq: Joule_heating} as
\begin{equation}
    I = \sqrt{\frac{P_{max}}{R}} = \sqrt{\frac{0.8}{30}} = 0.1633A
\end{equation}

\subsubsection{Constant Voltage vs Constant Current power supplying}
To power our coil we have two options, we can either provide a \textbf{constant voltage} or a \textbf{constant current}.

Using a constant current source is not advisable due to the \textbf{heating problem} of the coil.
At \textbf{high currents}, as the coil is run, it will \textbf{heat up} and \textbf{its resistance will increase which will induce the power source to increase the voltage supplied to keep the current constant}.
This in turn will cause the coil to heat up even more and the cycle will continue until the coil is damaged.
This is the phenomenon of thermal runaway we discussed before.

Instead, using a \textbf{constant voltage source} as the resistance increases due to the coil \textbf{exceeding the heating and power threshold we will only induce a decrease in the supplied current} which results in a loss of magnetic field strength but the coil won't get damaged.

