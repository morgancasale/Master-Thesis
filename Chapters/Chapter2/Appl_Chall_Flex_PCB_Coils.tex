\section{Application challenges of Flexible PCB coils}
Object pose estimation is a fundamental task in computer vision that involves determining the spatial orientation and position of an object in a given environment. The "pose" refers to the object's six degrees of freedom (6DOF), which include its translation (movement) along the x, y, and z axes and its rotation around these axes (roll, pitch, and yaw). The goal is to accurately understand how an object is positioned and oriented with respect of a reference coordinate system.

\begin{figure}[th]
    \centering
    \includegraphics[scale=0.5]{Figures/Chapter2/Yaw_Axis_Corrected.png}
    \caption[Roll, Pitch and Yaw rotations (RPY) of an object.]{Roll, Pitch and Yaw rotations (RPY) of an object.}
    \label{fig:RPY}
\end{figure}

Key aspects of object pose estimation include:
\begin{itemize}
    \item \textbf{Object Representation:} Objects are often represented by 3D geometric models or point clouds. These models describe the shape and structure of the object in a coordinate system.
    \item \textbf{Sensors:} Pose estimation relies on data acquired from sensors, such as cameras or depth sensors. Each sensor type has its strengths and limitations in capturing the necessary information for pose estimation.
    \item \textbf{Feature Extraction:} Features or keypoints (landmarks) are identifiable points on the object's surface that can be matched between the 3D model and the sensor data. These features serve as reference points for determining pose.
    \item \textbf{Matching and Correspondence:} The process involves finding correspondences between the features in the 3D model and those detected in the sensor data. Techniques such as feature matching and point cloud registration are employed for this purpose.
    \item \textbf{Pose Computation:} Once correspondences are established, algorithms calculate the pose parameters. This step involves identifying translation and rotation that best align the 3D model with the observed features in the sensor data.
    \item \textbf{Optimization:} Iterative optimization methods are used to refine the initial pose estimation. These methods aim to minimize the difference between predicted and observed feature locations.
    \item \textbf{Applications:} Object pose estimation is crucial in various applications, such as robotic manipulation, augmented reality, autonomous navigation, and quality control in manufacturing. It enables machines to interact with the environment and make informed decisions based on the perceived spatial relationships of objects.
\end{itemize}

\begin{figure}[th]
    \centering
    \includegraphics[scale=1]{Figures/Chapter2/Pose_estimation.png}
    \caption[Example of applications of pose estimation.]{Example of applications of pose estimation.}
    \label{fig:PoseEstimation}
\end{figure}

Accurate object pose estimation is essential for tasks where knowing an object's precise location and orientation is critical for effective and safe interaction with the environment. Advances in computer vision, machine learning, and sensor technologies contribute to the ongoing improvement of object pose estimation methods.


% -- Subsection 3.1
\subsection{Rise of high resistance}

% -- Subsection 3.2
\subsection{Parasitic effects due to AC current}

% -- Subsection 3.3
\subsection{Joule effect}

\subsubsection{Need for heat dissipation}
