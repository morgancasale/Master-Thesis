\subsection{Multi-layer PCB coils}
Another approach to miniaturizing PCB coils is to create \textbf{multi-layer coils}. 
This is done by stacking \textbf{multiple layers} of \textbf{PCBs} on top of each other, with each layer containing a different part of the coil. This allows for the creation of coils with a \textbf{higher number of windings} in a \textbf{same space}. The \textbf{main challenge} with multi-layer PCB coils is the \textbf{alignment of the different layers}. If the layers are not aligned properly, the coil will not function correctly.

Current manufacturing allows for up to 10 layers of PCBs to be stacked on top of each other. However, the more layers that are added, the more difficult it becomes to align the layers correctly.
If the layers are not aligned properly, the \textbf{magnetic field} generated by each layer will also \textbf{not be aligned}, which can lead to a \textbf{decrease in the efficiency} of the coil due to \textbf{interferences}.


\subsubsection{Total inductance}
\begin{samepage}
    Considering a two layers coil the total inductance can be calculated as:
    \nopagebreak

    \begin{equation}
        L_s = 2L_0 + 2M,   M = K_c \cdot L_0
    \end{equation}
    \nopagebreak

    Where:
    \begin{itemize}
        \item \( L_s \) is the total inductance of the coil [H].
        \item \( L_0 \) is the inductance of a single layer [H].
        \item \( M \) is the mutual inductance between the two layers [H].
        \item \( K_c \) is the coupling coefficient between the two layers.
    \end{itemize}
\end{samepage}

%\begin{samepage}
    Then $K_c$ can be calculated with an empirical formula derived from multiple measurements by \textit{Jonsenser Zhao} \cite{Multilayer_spiral_inductors} as
    %\nopagebreak

    \begin{equation}
        K_c = \frac{N^2}{0.64[(0.184d^3-0.525d^2+1.038d+1.001)(1.67N^2-5.84N+65)]}
    \end{equation}
    %\nopagebreak

    Where:
    \begin{itemize}
        \item \( N \) is the number of turns of the coil.
        \item \( d \) is the distance between the two layers [m].
    \end{itemize}
%\end{samepage}

\begin{samepage}
    \subsubsection{Magnetic field generated by a Multilayer coil}
    We can use equation \ref{eq: Inductance_&_flux} and $L_s$, calculated in the previous point, to find the magnetic flux through the coils
    \begin{equation}
        \Phi(I) = L_s \cdot I
    \end{equation}
    \nopagebreak

    Then with equation \ref{eq: Magnetic_flux_&_field} we can derive the total magnetic field as
    \begin{equation}
        B_t = \frac{L_s \cdot I}{\pi r^2}
    \end{equation}
    \nopagebreak

    Where:
    \begin{itemize}
        \item \( B_t \) is the total magnetic field [T].
        \item \( r \) is the radius of the coil [m].
    \end{itemize}
\end{samepage}

% TODO: Maybe insert a graph for the effect of the mutual inductance 


